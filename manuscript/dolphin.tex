% Preamble
\documentclass[12pt,a4paper]{article}
\usepackage{enumerate} 	
\usepackage{setspace}						
\usepackage{authblk}	
\usepackage{graphicx} 	
%\usepackage[nomarkers, nolists]{endfloat} 
\usepackage{pdflscape}	
\usepackage{mathtools}	
\usepackage[osf]{mathpazo} 
\usepackage{fixltx2e}	
\usepackage{lineno} 
\usepackage{ms}    	
\usepackage{hyperref}
\usepackage[round]{natbib} 
\usepackage{setspace}

\setcounter{secnumdepth}{0} 
\raggedright 			
\pagenumbering{arabic}	
\linenumbers

% Title page information

\title{Quantifying convergence in river dolphins}

\author{
	Charlotte E Page$^{1,2*}$ and Natalie Cooper$^{2}$
}

\date{}

\affiliation{\noindent{\footnotesize
	$^1$ Imperial\\ 
	$^2$ NHM\\
	$^*$Corresponding author\\
}}

\vfill

\runninghead{Convergence in river dolphins}
\keywords{river dolphins, geometric morphometrics, morphology} 

\begin{document}

\mstitlepage
\parindent = 1.5em
\addtolength{\parskip}{.3em}

\section{Abstract}

	

\newpage
\raggedright
\doublespacing
\setlength{\parindent}{1cm}

\section{Introduction}

	
	
\section{Materials and Methods}

	

\section{Results} 
	
%PCA figure
	%\begin{figure}[!htbp]
	%\centering
	%\includegraphics[width=1\linewidth, height=1\textheight, keepaspectratio]{figures/PCA_threeplot.png}
	%\caption[Morphospace (principal components) plot of morphological diversity in tenrec and golden mole skulls.]
	%	{Principal components plots of the morphospaces occupied by tenrecs (triangles, n=31 species) and golden moles (circles, n=12 species) for skulls in dorsal (top left), ventral (top right) and lateral (bottom left) views. Each point represents the average skull shape of an individual species. Axes are principal component 1 (PC1) and principal component 2 (PC2) of the average scores from principal components analyses of mean Procrustes shape coordinates for each species.}
	%\label{fig:threePCA}
	%\end{figure}


% Results table
	%\begin{table}[!htbp]			
		%\caption[Comparing morphological diversity in tenrecs and golden moles.]
		%{Morphological diversity in tenrecs compared to golden moles (12 species). N is the number of tenrec species: 31 species or 17 species including just five representatives of the \textit{Microgale} Genus. Morphological diversity of the Family is the mean Euclidean distance from each species to the Family centroid. Significant differences between the two Families (p$<$0.05) from two-tailed t-tests are highlighted in bold.}
		%\input{tables/diversity.results} 
		%\label{tab:diversity}  
	%\end{table}
	

\section{Discussion}


\section{Conclusions}
	

\section{Acknowledgments}
	We thank Richard Sabin (Natural History Museum, London) for help accessing the collections.
	
%\bibliographystyle{sysbio} 
%\bibliography{dolphin-refs} 

\end{document}